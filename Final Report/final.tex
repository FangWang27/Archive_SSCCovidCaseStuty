\documentclass{report}
\usepackage{amsmath}
%\usepackage{amssymb} # it does not work with MnSymbol
\usepackage{amsthm,mathrsfs}
\usepackage{enumitem}
\usepackage[top=1in, bottom=1in, left=1in, right=1in]{geometry}
\usepackage{bm}
\usepackage{physics}
\usepackage{multirow}
\usepackage{tabularx}
\usepackage{graphicx}
\usepackage{hyperref}
\usepackage{enumitem}
\usepackage{booktabs}
\usepackage{cite}
\usepackage{algorithm}
\usepackage{algpseudocode}
\usepackage{cprotect}
\usepackage{graphicx}
\graphicspath{ {figures/} }
%Font used for low resolution printer, i.e 
\usepackage[no-math]{fontspec}
\usepackage{fourier}
\usepackage{MnSymbol}
\setmainfont{Linux Libertine O} 
%------------------------------
\author{Fang Wang}
\title{ \textbf{\huge{Covid Project}}} 

\begin{document}
\newcommand{\RR}{\mathbb{R}}
\newcommand{\ZZ}{\mathbb{Z}}
\newcommand{\NN}{\mathbb{N}}
\newcommand{\QQ}{\mathbb{Q}}
\newcommand{\CC}{\mathbb{C}}
\newcommand{\e}{\mathrm{e}}
\newcommand{\sumi}[2][1]{\sum\limits_{i=#1}^{#2}}
\newcommand{\sumk}[2][1]{\sum\limits_{k=#1}^{#2}}
\newcommand{\sumj}[2][1]{\sum\limits_{j=#1}^{#2}}
\newcommand{\sumx}[2][1]{\sum\limits_{x=#1}^{#2}}
\newcommand{\sumn}[2][1]{\sum\limits_{n=#1}^{#2}}
\newcommand{\prodi}[2][1]{\prod\limits_{i=#1}^{#2}}
\newcommand{\prodk}[2][1]{\prod\limits_{k=#1}^{#2}}
\newcommand{\prodj}[2][1]{\prod\limits_{j=#1}^{#2}}
\newcommand{\E}[2][]{ \mathbb{E}_{#1} \left[ #2 \right]}
\newcommand{\Var}[1]{ \mathrm{Var} \left[ #1 \right]}
\newcommand{\Cov}[1]{ \mathrm{Cov} \left[ #1 \right]}
\renewcommand{\P}[2][]{ \mathbb{P}_{#1} \left( #2 \right)}
\newcommand{\iidis}{\stackrel{iid}{\sim}}
\newcommand{\toP}{\stackrel{P}{\to}}
\newcommand{\toD}{\stackrel{\mathrm{D}}{\to}}
\newcommand{\eqD}{\stackrel{\mathrm{D}}{=}}
\newcommand{\toas}{\stackrel{\mathrm{a.s}}{\to}}
\newcommand{\I}[1]{\mathbb 1_{\{#1\}}}
\newtheorem{thm}{Theorem}[chapter]
\newtheorem{prop}{Proposition}[chapter]
\newtheorem{lem}{Lemma}[chapter]
\newtheorem{cor}{Corollary}[chapter]

\theoremstyle{definition}
\newtheorem{defn}{Definition}[chapter]
\newtheorem{eg}{Example}[chapter]

\theoremstyle{remark}
\newtheorem*{rem}{Remark}
\maketitle

\chapter{Introduction} \label{chapter intro}

More than fifteen thousands of deaths in Canada are linked to the ongoing pandanmic caused by the coronavirus disease 2019 (COVID-19) \cite{dong2020interactive}, many of which are the residents
in long term care center (LTH). 



\chapter{Data and Methods} \label{chapter methods}





\bibliographystyle{unsrt}
\bibliography{ref}
\end{document}